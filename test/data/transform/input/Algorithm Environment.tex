\documentclass{article}

\newenvironment{steps}%
{%
\vspace{0.1cm}%
\begin{adjustwidth}{0.3cm}{0cm}%
\begin{description}%
\setlength{\itemsep}{0pt}%
\setlength{\parskip}{0pt}%
\setlength{\parsep}{0pt}%
}
{%
\end{description}%
\end{adjustwidth}%
\vspace{0.1cm}%
}

\newcounter{algorithm}[section]
\renewcommand{\thealgorithm}{\arabic{section}.\arabic{algorithm}}%
\newenvironment{algorithm}[1][]
{\refstepcounter{algorithm}%
 \par%
 \vspace{.5\baselineskip\@plus.2\baselineskip\@minus.2\baselineskip}% Space above
\noindent\textbf{Algorithm~\thealgorithm}\ifx&#1&\else~(#1)\fi\textbf{.}%
 %\noindent\textbf{Algorithm}\ifx&#1&\else~(#1)\fi\textbf{.}%
 \begin{steps}}%\begin{algorithm}
{\end{steps}%
%\vspace{.5\baselineskip\@plus.2\baselineskip\@minus.2\baselineskip}% Space below
}% \end{algorithm}

\begin{document}

\dots

\section{Nonuniform algorithm for explicit 3POL}%
\label{sec:algo:explicit:nonuniform}

We begin with the description of a nonuniform algorithm for explicit 3POL which
we use later as a basis for other algorithms. We prove the following:
\begin{theorem}\label{thm:explicit:act}
	There is a bounded-degree ADT of depth
	$O(n^{12/7+\varepsilon})$
	for explicit 3POL\@.
\end{theorem}

\paragraph{Idea}
We partition the sets $A$ and $B$ into small groups of consecutive
elements. That way, we can divide the $A\times B$ grid into cells with the
guarantee that each curve $c = f(x,y)$ intersects a small number
of those cells. For each such curve and each cell it intersects, we
search $c$ among the values $f(a,b)$ for all $(a,b)$ in a given intersected
cell. We generalize Fredman's trick~\cite{F76} --- and how it is used in Gr\o
nlund and Pettie's paper~\cite{GP14} --- to quickly obtain a sorted order on
those values, which provides us a logarithmic search time for each cell.
Below is a sketch of the algorithm.

\begin{algorithm}[Nonuniform algorithm for explicit 3POL]\label{algo:ne}
\item[input] $A = \{\,a_1 < \cdots < a_n\,\},B = \{\,b_1<\cdots<b_n\,\},
    C = \{\,c_1<\cdots<c_n\,\}
    \subset \mathbb{R}$.
\item[output] \emph{accept} if $\exists\, (a,b,c) \in A \times B \times C$ such that $c
    = f(a,b)$, \emph{reject} otherwise.%
    \footnote{Note that it is easy to modify the algorithm to count or report the
    solutions. In the latter case, the algorithm becomes output sensitive.}
\item[1.] Partition the intervals $[a_1,a_n]$ and $[b_1,b_n]$ into blocks
    $A_i^*$ and $B_j^*$ such that $A_i = A \cap A_i^*$ and $B_j = B
    \cap B_j^*$ have size $g$.
\item[2.] Sort the sets $f(A_i \times B_j) = \{\, f(a,b) \st (a,b) \in A_i
    \times B_j\,\}$ for all $A_i,B_j$. This is the only step that
    is nonuniform.
\item[3.] For each $c \in C$,
\item[3.1.] For each cell $A_i^* \times B_j^*$ intersected by the curve
$c=f(x,y)$,
\item[3.1.1.] Binary search for $c$ in the sorted set $f(A_i \times B_j)$.
If $c$ is found, \emph{accept} and halt.
\item[4.] \emph{reject} and halt.
\end{algorithm}
%
Like in Gr\o nlund and Pettie's $\tilde{O}(n^{3/2})$ decision tree for
3SUM~\cite{GP14}, the key is to give an efficient implementation
of step~\textbf{2}.

\dots

\end{document}
